%% LyX 2.2.3 created this file.  For more info, see http://www.lyx.org/.
%% Do not edit unless you really know what you are doing.
\documentclass[12pt,british]{IEEEtran}
\usepackage[T1]{fontenc}
\usepackage[latin9]{inputenc}
\setcounter{tocdepth}{2}
\usepackage{color}
\usepackage{babel}
\usepackage{array}
\usepackage{float}
\usepackage{calc}
\usepackage{multirow}
\usepackage[unicode=true,
 bookmarks=true,bookmarksnumbered=true,bookmarksopen=false,
 breaklinks=false,pdfborder={0 0 1},backref=false,colorlinks=true]
 {hyperref}
\hypersetup{pdftitle={The LyX User's Guide},
 pdfauthor={LyX Team},
 pdfsubject={LyX},
 pdfkeywords={LyX},
 linkcolor=black, citecolor=black, urlcolor=blue, filecolor=blue, pdfpagelayout=OneColumn, pdfnewwindow=true, pdfstartview=XYZ, plainpages=false}

\makeatletter

%%%%%%%%%%%%%%%%%%%%%%%%%%%%%% LyX specific LaTeX commands.
%% Because html converters don't know tabularnewline
\providecommand{\tabularnewline}{\\}

%%%%%%%%%%%%%%%%%%%%%%%%%%%%%% Textclass specific LaTeX commands.
\newlength{\lyxlabelwidth}      % auxiliary length 

%%%%%%%%%%%%%%%%%%%%%%%%%%%%%% User specified LaTeX commands.
% DO NOT ALTER THIS PREAMBLE!!!
%
% This preamble is designed to ensure that the User's Guide prints
% out as advertised. If you mess with this preamble,
% parts of the User's Guide may not print out as expected.  If you
% have problems LaTeXing this file, please contact 
% the documentation team
% email: lyx-docs@lists.lyx.org

\usepackage{ifpdf} % part of the hyperref bundle
\ifpdf % if pdflatex is used

 % set fonts for nicer pdf view
 \IfFileExists{lmodern.sty}{\usepackage{lmodern}}{}

\fi % end if pdflatex is used

% for correct jump positions whe clicking on a link to a float
\usepackage[figure]{hypcap}

% the pages of the TOC is numbered roman
% and a pdf-bookmark for the TOC is added
\let\myTOC\tableofcontents
\renewcommand\tableofcontents{%
  \frontmatter
  \pdfbookmark[1]{\contentsname}{}
  \myTOC
  \mainmatter }

% macro for italic page numbers in the index
\newcommand{\IndexDef}[1]{\textit{#1}}

% for customized page headers/footers
% only needed because they are only used in one section of the document
\usepackage{fancyhdr}
% change header rule width
\renewcommand{\headrulewidth}{2pt}

% used to have extra space in table cells
\@ifundefined{extrarowheight}
 {\usepackage{array}}{}
\setlength{\extrarowheight}{2pt}

% workaround for a makeindex bug,
% see sec. "Index Entry Order"
% only uncomment this when you are using makindex
%\let\OrgIndex\index 
%\renewcommand*{\index}[1]{\OrgIndex{#1}}

\makeatother

\begin{document}
\onecolumn

\title{Laboratory Exercise 2}

\IEEEspecialpapernotice{ELEN4020, 16 March 2018}

\author{Shane House (749524), Benjamin Rosen (858324)}
\maketitle

\section{Algorithm Description}

\noindent\fbox{\begin{minipage}[t]{1\columnwidth - 2\fboxsep - 2\fboxrule}%
\[
\left(\begin{array}{cccccc}
d & x & x & x & x & x\\
\mathbf{1} & d & x & x & x & x\\
\mathbf{2} & \mathbf{3} & d & x & x & x\\
\mathbf{4} & \mathbf{5} & \mathbf{6} & d & x & x\\
\mathbf{7} & \mathbf{8} & \mathbf{9} & \mathbf{10} & d & x\\
\mathbf{11} & \mathbf{12} & \mathbf{13} & \mathbf{14} & \mathbf{15} & d
\end{array}\right)
\]

\begin{equation}
1;~3;~6;~10;~15;~...\label{eq:numSeq}
\end{equation}

\begin{equation}
T_{n}=0.5n^{2}+0.5n\label{eq:quadNum}
\end{equation}

\begin{equation}
n=0.5+\sqrt{0.25+2T_{n}}\label{eq:rearrangedQuad}
\end{equation}

\begin{equation}
x=floor(n)\label{eq:x}
\end{equation}

\begin{equation}
y=trunc((n-x)*(x+1))\label{eq:y}
\end{equation}
%
\end{minipage}}

\medskip

\subsection{Mathematical Explanation\label{subsec:Mathematical-Explanation}}

An NxN matrix, with N set to 6, is shown above. \textit{d} represents
the diagonal values that do not need to be swapped. The numbers in
bold indicate the swap iteration, while the \textit{x}'s are the values
to be swapped with correspondingly.

The goal of this algorithm is to evenly divide the number of swaps
among the available threads. To this end, a method needed to be developed
that mapped the swap iterations to an \textit{(x, y) }position in
the matrix. Looking at the matrix above, the last swap iteration value
in every row forms a quadratic number pattern, shown in Equation \ref{eq:numSeq}.
The equation that relates the value of the pattern ($T_{n}$) to the
iteration \textit{n} is given by Equation \ref{eq:quadNum}. This
is a quadratic equation which can be solved using the quadratic formula.
Since the iteration value cannot be negative, only one of the values
is used. The simplification of the quadratic formula is depicted in
Equation \ref{eq:rearrangedQuad}. From this value, \textit{x }and
\textit{y }values can be obtained using Equations \ref{eq:x} and
\ref{eq:y}. For values in between the ones given in the number series,
Equation \ref{eq:rearrangedQuad} will give a number in between the
iteration value. For example, swap iteration 5 will give an \textit{n}
value of 3.7, which indicates it is between the 3\textsuperscript{rd}
and 4\textsuperscript{th} iterations. Therefore, we know it is in
the 3\textsuperscript{rd} row of the matrix. It was found that the
value of \textit{n} increases for each iteration depending on the
number of values in that row. Thus, the \textit{y }value is determined
by taking the decimal place, of 0.7 in the above example, and multiplying
it by the number of values in that row (3 for our case). Taking only
the integer value gives us the column, 2 for the example.

\subsection{Program Implementation}

\subsubsection{Pthread}

A main thread delegates the swaps to child threads by first determining
the number of swaps each thread will perform. It then iterates through
the start positions (increasing by the number of swaps) and determines
the child thread's start position (using the algorithm outlined in
Section \ref{subsec:Mathematical-Explanation}). This value, along
with a reference to the matrix and the number of swaps to perform,
is passed in a struct to the thread. Each thread performs the swaps
by cycling through the swap iterations (as in the matrix above) and
then joins the main thread. The main thread, after delegating the
work, performs the last batch of swaps.

\subsubsection{OpenMP}

A parallel region is defined, in which the threads, including the
main thread, perform the swaps as in the above section. However, each
thread determines its own start position, by making use of its thread
ID.

\subsubsection{Timing}

For accurate timing of parallel threads, \textit{clock\_gettime()}
is used, which uses the actual time (instead of CPU cycles). This
delivered more consistent timing when running the program multiple
times.

\section{Results}

Table \ref{tab:Results} summarises the times for the different dimensions,
threads and implementations used. The general trend was that for small
dimensions, the timings varied significantly but the smaller number
of threads tended to have the better times. For large N, having more
threads caused the timings to improve. OpenMP, except for some outliers,
completed the operations quicker than Pthread. 

Of course, only a certain number of threads could be executed in parallel
on our machines, meaning that the time benefit of having more threads
was not significant.

\begin{table}[H]
\caption{Results for Pthread and OpenMP Programs for Varying Matrix Dimensions
and Threads\label{tab:Results}}
\centering{}%
\begin{tabular}{|c|c|c|c|c|}
\hline 
Implementation & No. of Threads & $N_{0}=N_{1}=128$ & $N_{0}=N_{1}=1024$ & $N_{0}=N_{1}=8192$\tabularnewline
\hline 
\hline 
\multirow{6}{*}{OpenMP} & 4 & 0.0213945 & 0.0146729 & 0.5200562\tabularnewline
\cline{2-5} 
 & 8 & 0.0004815 & 0.0143658 & 0.5099768\tabularnewline
\cline{2-5} 
 & 16 & 0.0011964 & 0.0074405 & 0.5125988\tabularnewline
\cline{2-5} 
 & 32 & 0.0018648 & 0.0100943 & 0.5074549\tabularnewline
\cline{2-5} 
 & 64 & 0.0037445 & 0.0084525 & 0.5076541\tabularnewline
\cline{2-5} 
 & 128 & 0.0179112 & 0.0137622 & 0.5088798\tabularnewline
\hline 
\hline 
\multirow{6}{*}{Pthread} & 4 & 0.0006303 & 0.0078572 & 0.7830937\tabularnewline
\cline{2-5} 
 & 8 & 0.0008100 & 0.0099369 & 0.8439945\tabularnewline
\cline{2-5} 
 & 16 & 0.0017638 & 0.0110775 & 0.8351318\tabularnewline
\cline{2-5} 
 & 32 & 0.0034362 & 0.0115999 & 0.8432556\tabularnewline
\cline{2-5} 
 & 64 & 0.0072569 & 0.0142217 & 0.8368004\tabularnewline
\cline{2-5} 
 & 128 & 0.0139730 & 0.0193327 & 0.8371888\tabularnewline
\hline 
\end{tabular}
\end{table}

\end{document}
