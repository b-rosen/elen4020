%% LyX 2.2.3 created this file.  For more info, see http://www.lyx.org/.
%% Do not edit unless you really know what you are doing.
\documentclass[12pt,british]{IEEEtran}
\usepackage[T1]{fontenc}
\usepackage[latin9]{inputenc}
\setcounter{tocdepth}{2}
\usepackage{color}
\usepackage{babel}
\usepackage{float}
\usepackage[unicode=true,
 bookmarks=true,bookmarksnumbered=true,bookmarksopen=false,
 breaklinks=false,pdfborder={0 0 1},backref=false,colorlinks=true]
 {hyperref}
\hypersetup{pdftitle={The LyX User's Guide},
 pdfauthor={LyX Team},
 pdfsubject={LyX},
 pdfkeywords={LyX},
 linkcolor=black, citecolor=black, urlcolor=blue, filecolor=blue, pdfpagelayout=OneColumn, pdfnewwindow=true, pdfstartview=XYZ, plainpages=false}

\makeatletter

%%%%%%%%%%%%%%%%%%%%%%%%%%%%%% LyX specific LaTeX commands.
\floatstyle{ruled}
\newfloat{algorithm}{tbp}{loa}
\providecommand{\algorithmname}{Algorithm}
\floatname{algorithm}{\protect\algorithmname}

%%%%%%%%%%%%%%%%%%%%%%%%%%%%%% Textclass specific LaTeX commands.
\newlength{\lyxlabelwidth}      % auxiliary length 

%%%%%%%%%%%%%%%%%%%%%%%%%%%%%% User specified LaTeX commands.
% DO NOT ALTER THIS PREAMBLE!!!
%
% This preamble is designed to ensure that the User's Guide prints
% out as advertised. If you mess with this preamble,
% parts of the User's Guide may not print out as expected.  If you
% have problems LaTeXing this file, please contact 
% the documentation team
% email: lyx-docs@lists.lyx.org

\usepackage{ifpdf} % part of the hyperref bundle
\ifpdf % if pdflatex is used

 % set fonts for nicer pdf view
 \IfFileExists{lmodern.sty}{\usepackage{lmodern}}{}

\fi % end if pdflatex is used

% for correct jump positions whe clicking on a link to a float
\usepackage[figure]{hypcap}

% the pages of the TOC is numbered roman
% and a pdf-bookmark for the TOC is added
\let\myTOC\tableofcontents
\renewcommand\tableofcontents{%
  \frontmatter
  \pdfbookmark[1]{\contentsname}{}
  \myTOC
  \mainmatter }

% macro for italic page numbers in the index
\newcommand{\IndexDef}[1]{\textit{#1}}

% for customized page headers/footers
% only needed because they are only used in one section of the document
\usepackage{fancyhdr}
% change header rule width
\renewcommand{\headrulewidth}{2pt}

% used to have extra space in table cells
\@ifundefined{extrarowheight}
 {\usepackage{array}}{}
\setlength{\extrarowheight}{2pt}

% workaround for a makeindex bug,
% see sec. "Index Entry Order"
% only uncomment this when you are using makindex
%\let\OrgIndex\index 
%\renewcommand*{\index}[1]{\OrgIndex{#1}}

\makeatother

\usepackage{listings}
\renewcommand{\lstlistingname}{Listing}

\begin{document}
\onecolumn

\title{Laboratory Exercise 1}

\IEEEspecialpapernotice{ELEN4020, 23 February 2018}

\author{Shane House (749524), Benjamin Rosen (858324)}
\maketitle

\section{OpenMP}

When the sample code was run, it was found that 4 ``hello world''s
were printed. OpenMP runs the code in a number of parallel processes,
which defaults to the number of threads capable by the machine. The
code was run on a 2 core machine, with each core capable of 2 threads,
which is why it printed 4 times.

\section{Lab}

The N-dimensional array was constructed using one single array. This
is because a multi-dimensional array with dynamic dimension cannot
be passed to a function. In addition, the computer stores a multi-dimensional
array in contiguous memory, so it is effectively a 1D array.

First, a helper function was declared, which calculated the total
length of the 1D array by multiplying the dimensions together.

For the first function, a simple for loop was used that runs through
the entire 1D array and sets all values to 0.

The second function multiplies the array length by 0.1 (i.e. 10\%),
and uses that value as an increment in the for loop. The for loop
sets each value that it encounters to 1. This ensures that there is
a uniform distribution of 1s throughout the array.

The final function multiplies the array length by 0.05 (5\%) to obtain
the number of elements to check. For each iteration, a new random
position in the linear (1D) array is selected and its value printed.
The actual position in the multi-dimensional array is calculated using
Algorithm \ref{alg:multi-dim}.

\begin{algorithm}
\caption{Method to Calculate Multi-Dimensional Position from Linear Array Position\label{alg:multi-dim}}

\begin{lstlisting}[tabsize=2]
Begin Algorithm:
	linearPos = linear array position
	for each dimension, working backwards, do:
		dimensionLength = length of dimension
		position for multi-dimensional array = linearPos mod dimensionLength
		linearPos /= dimensionLength
	end for
End Algorithm
\end{lstlisting}
\end{algorithm}

A for loop prints out the multi-dimensional position to console. The
reason we worked backwards, by generating a random number and then
working out the N-dimensional position, was to ensure that the distribution
would be uniform for the \textbf{entire }array. For the other method,
where a random number is determined for each dimension and the linear
position calculated after, the distribution will only occur over each
dimension, but not necessarily the entire N-dimensional array.
\end{document}
